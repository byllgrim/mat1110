\paragraph{Newtons metode i matlab} \mbox{} \\
\lstinputlisting[caption=10 iterasjoner med matlab]{oppg2c.m}
\begin{lstlisting}[caption=Output]
$ octave oppg2c.m
x =

 Columns 1 through 8:

    1.0000    3.0000    3.0000    3.0000    3.0000    3.0000    3.0000    3.0000
    1.0000    5.0000   25.0000   15.2439   10.8143    9.2607    9.0071    9.0000

 Columns 9 and 10:

    3.0000    3.0000
    9.0000    9.0000
\end{lstlisting}

Den konvergerer det ene av nulltpunktene som jeg fant i oppgave 2a, (3,9).



\paragraph{Alternativt startpunkt} \mbox{} \\
Hva skjer hvis $x_0 = (1,0.1)$.

\begin{lstlisting}[caption=Output]
$ octave oppg2c-alt.m
x =

 Columns 1 through 8:

   1.00000   3.00000   3.00000   3.00000   3.00000   3.00000   3.00000   3.00000
   0.10000  -0.51250  -0.02620  -0.00008  -0.00000  -0.00000   0.00000   0.00000

 Columns 9 and 10:

   3.00000   3.00000
   0.00000   0.00000
\end{lstlisting}

Nå fant vi det andre nullpunktet fra 2a, $(3,0)$.
